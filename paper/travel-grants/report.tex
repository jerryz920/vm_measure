\documentclass{article}

\begin{document}

\title{Travel Report for APSys'11 at Shanghai}
\author{Mingliang Liu}
\date{Thu Jul 14,  2011}

\maketitle

\par
Firstly, I presented the slides to find my poor spoken English. Maybe the
spoken is redundant here. The most exciting experience was the preparation of
the slides. I wrote the draft version about 3 weeks ago using Latex (Next time
I'll use PPT for nice co-operation). Then I tried my best to refine it.  Prof.
Ma told me to talk it to the camera of my laptop at least 5 times before the
workshop. I followed her suggestion and it did help me a lot. This was the
only way I made myself familiar with my talk. Prof. Ma thought a serial of
slides with few or no pictures was unacceptable. Thus, I added about 7 pages
of picture to show the examples. Chen Sir spent about two hours to hear my
talk and gave me very useful feedback one slide by another. The most impressed
comments was the logic. Yes, the logic. Chen Sir asked me to make it clear and
logical from one line of a slide to the next one. To make it easy, he
reorganized the logic of the talk for me producing \~10 items of
refinements.  As to the slides order, I followed to the paper's. Yan Zhai read
all the suggestions and helped me a lot to implement them. Jidong and Dehao
listened my talk at the eve and provided tons of useful feedback. This was a
huge education for me. I think more experience alike must happen to every
qualified PhD.

\par
Second, I tried myself to make new friends the other day. After the talk I
given, I found it much easier to chat up. I talked to a guy from EMC who's
interested in our work and we exchanged the contact info using Barcode Scanner
:-P I then met two guys discussing together, one of whom was the classmate of
Chen Sir, and the other of whom was the classmate of Prof.  Ma. Small word.
Yunyun Jiang and I joined them. They proposed one good question to me: What
are the top considerations if you are supposed to design a file system to make
HPC applications run in the cloud? I told him that the question was so
distinguished and challenging that I have no good answer yet.  I tried to
analyze the difference between the traditional HPC facilities and the cloud
platform. I then talked to a guy from Microsoft and he told me to try to find
more cloud platforms besides Amazon which are dedicated for HPC runs. I don't
remember his recommendations but I know now there must be some guy in another
place investigating the HPC in cloud. I also joined the Peter and Jianian
Yan's talk when I picked up my drink.

\par
Last but not least, my final talk was devoted to Gernot. He is the boss of Yao
Shi at Australia (I knew it later). We seated together in the bus to the World
Expo. I asked him what's his expectation for his PhD candidates. He told me
that at least one paper was published in EurSys, OSDI, USENIX and SOSP. More
are better. He also asked my previous publish list and I was ashamed to say
no. I invited him to pay a visit to our group next time he goes to Beijing.

\par
To conclude, I think it's an amazing experience for me to attend an
international workshop like APSys. Research is funny. 
\end{document}
