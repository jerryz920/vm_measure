\section{Memory Allocator}
	\label{sec:alloc}
Memory allocator is a major interface that operating systems exposed to the
users. On Linux, users request memory throught the system call \emph{brk} or
use the memory mapped file through \emph{mmap}. This can also be unified by
standard library interface \emph{malloc}.  The operating system, will ensure
that if the requests are granted, then visiting these address would not cause
protection errors and thus can be used by user applications. However, this does
not mean operating system will allocate the physical page, say, setting up the
virtual page to physical mapping immediately.  Rather than fit user's request
once and for all, operating systems may employ a lazy allocation strategy. In
this configuration, only when a page fault occurs, will the operating system
check the need to allocate physical memory. In this section, we verify two things.
First is that whether operating system uses on-demand allocation, and second
is when will operating system zero out the pages for security concerns.

\subsection{Methodology}
The strategy I used is quite simple. To determine whether operating systems
allocate pages on demand, we just allocate a large memory area, and visit the
performance we walk upon first time. It should be quite different than normal
walking. To further distinguish what operating systems are doing, I will measure
the time with and without allocating system call, here I used \emph{mmap}, as
well as the normal walk time.

To check when does operating system zero out the pages, if the allocation is
not done at the system call time, then it's only possible for operating system
either to 1.) choose a known zero page 2.) zero out page on demand. In order
to exclude or include the first approach, I measured the time of allocation
both when free memory is abundant and is scarse. I try to make sure that 
operating system will not have many free zero pages to borrow, then I could
figure out whether there will be extra cost during allocation. One problem may
still remain is that kernel can zero out pages when pages are unmapped. So I
also measure the normalized time used to unmap a clean page and a dirty page.

\subsection{Experiments}




\subsection{Discussion}
One thing hard to illustrate is whether operating system will 'smartly'
allocate some pages for small requests. For example, when user requests for
only 1 page, then allocate the page immediately to avoid a page fault can be
somewhat a good choice. This behavior should be able to get observed from
the hardware monitoring, but due to both the time limit and space limit, I
didn't implement this as part of the benchmark.
