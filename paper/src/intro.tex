\section{Introduction}
    \label{sec:intro}
    %PVFS\cite{bib:pvfs}, and Lustre~\cite{lustre}.  For a selected parallel
    %\begin{itemize} 
    %    \item 
    %    \item 
    %    \item 
    %\end{itemize}
	
The virtual memory subsystem has become an indispensable constitution in modern
operating systems. With the proper hardware support in paging and segmentation,
operating systems build their own mechanisms to protection and abstractions to
users. This greatly simplifies the price to write correct code, and also make
the operating system more reliable to the user faults.

However, writing correct code is not equal to writing good code. The
applications may not perform well without knowing underlying operating systems
and hardware. For example, designing a good web server will often require good
knowledge of how to maximize the usage of memory and avoid long latency from
disks~\cite{c10k}. Thus, it would be quite attractive to reveal what is beneath
the beautiful illusory operating systems provide with you.

The virtual memory subsystem is quite huge, we mainly focus on several topics:
\begin{itemize}
\item TLB(Translation Lookaside Buffer), the 'cache' of the virtual memory. It
will be necessary to know how the TLB are working, how large it is. We try to
measure the TLB size through a set of experimentations, in order to know more
deep about this small buffer.
\item Huge Pages. To avoid TLB missing, huge pages can serve quite good for
this purpose. But while it has many benefits to use huge pages, what is the
cost? In this paper, I try to illustrate the cost of huge pages with the 
performance cost in page preparation, allocation aspects.
\item Memory Allocator, like \emph{malloc}, \emph{mmap}. The only thing users
can see is these allocators will allocate the virtual pages when invoked. However,
when can user really uses this page? What is the allocation policy of the physical
pages? These are all interesting questions to answer.
\item Optimization of Virtual Memory System. There are many optimization for the 
virtual memory, like better page replacement algorithm, prefetching, and so on. In
this paper, I explored the benefits from the object sharing.
\end{itemize}

To flexibly employ all kinds of measurement strategies, we choose Linux as our
major experiment environment. Four major experiments, and several minor ones
are carried out toward above topics, and most of them run well, matching my
understanding to the architecture and system. One thing that causes trouble is
the TLB. The measurement of TLB is not quite accurate, and I will explain the
reasons with the gathered timing and hardware events data.

The paper is organized in following way: section~\ref{sec:timing} will
introduce the environment, including time function I used, and some
configuration details. section~\ref{sec:tlb} is measuring different level TLB sizes.
section~\ref{sec:alloc} describes how we measure the memory allocator. section~\ref{sec:hugepage} will test
the huge page overhead, section~\ref{sec:sharedobj} illustrates the benefits from shared objects.

