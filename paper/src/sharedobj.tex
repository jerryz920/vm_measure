\section{Shared Code Page}
	\label{sec:sharedobj}
Shared code pages is an important optimization to the virtual memory. Shared code
usually appear as shared library or dynamic linking library in the system.
With shared library, code size are largely reduced.  Moreover, the shared
library is usually place independent and can be dynamically loaded by the
applications.  This enables great flexibility in design. For instance,
apache~\cite{apache} uses dynamic libraries to implement its modules, which can
be loaded on demand and built separately. In this section I compare the typical
applications' code size by linked them statically and dynamically. In addition,
I will present a simple web testing on the apache server compiled statically
and dynamically, to see if there is a performance difference.

\subsection{Methodology}
It's quite straightforward to measure the code size. I compiled
LLVM~\cite{llvm} and Apache~\cite{apache} to see the code size difference for
shared library version and statically linked version.

For web testing, I use the apachebench~\cite{apachebench} to fetch different
size of static web pages, including synthetic pages and home pages of big
sites. The MPM model I intentionally choose prefork model, since during
forking, process's address space will be copied. The statically linked code,
might have a little performance superior in external function call and
variables access~\cite{linker}. But when fork executes frequently, the code
size then destroy the tiny benefits.

\subsection{Experiments}
\subsubsection{Code Size}

\subsubsection{Performance Difference in Real Webserver}



\subsection{Discussion}
Originally I would like to compile firefox~\cite{firefox} to see the code size
change. But after some attempts I found they do not support static linking in their
build scripts so I moved to LLVM.

Although shared objects are extensively used in current systems, it is not
correct to say that static library is no longer required. Some problems 
raised only in shared objects, like library dependency, and the binary 
compatibility~\cite{dllhell}. Also, in scenario requires extreme good performance,
like high performance scientific computing, static library would be their
first choice.
