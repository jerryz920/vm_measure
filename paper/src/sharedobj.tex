\section{Shared Code Page}
	\label{sec:sharedobj}
Shared code pages is an important optimization to the virtual memory.
Previously I have shown the benefits from demand allocation as an optimization,
now I will go on exploring the code sharing. Shared code usually appears as a
shared library or say dynamic linking library. With shared library, code size
are largely reduced. Moreover, shared libraries are usually place independent
and can be dynamically loaded by the applications. This enables great
flexibility in design. For instance, apache~\cite{apache} uses dynamic
libraries to implement its modules, which can be loaded on demand and built
separately. In this section I will focus on comparing the typical applications'
code size by linked them statically and dynamically.

\subsection{Methodology}
It's quite straightforward to measure the code size. I compiled
LLVM~\cite{llvm} and Apache~\cite{apache} to see the code size difference for
shared library version and statically linked version.

I also tried to use an apache server with prefork mechanism, to compare the
performance of shared code and static code. But finally I gave up, it is
because the \emph{fork} system call on Linux will make parent-children sharing
the same code region, then the difference in code size will not be small. It is
possible to synthesis some benchmark, but it is not quite persuading, since it 
must reflect the pattern how real applications will use shared library Due to
these considerations, I didn't finish the experiments on this. 

\subsection{Experiments}

\begin{table}[ht] \scriptsize
\begin{tabular}{|l|c|c||l|c|c|}
 \hline
 name & shared & static & name & shared & static \\
 \hline
httpd             &     1.6M & 7.3M &  bugpoint          &     3.5M & 75M   \\
clang             &     283M & 459M &   clang-check       &     135M & 152M   \\
clang-tblgen      &     6.0M & 6.0M &   llc               &     908K & 146M   \\
lli               &     844K & 103M &   llvm-ar           &     484K & 14M   \\
llvm-as           &     248K & 17M  &  bcanalyzer   &     564K & 2.6M   \\
llvm-config       &     2.1M & 2.1M &   llvm-cov          &     132K & 2.2M   \\
llvm-diff         &     1.3M & 15M  &  llvm-dis          &     384K & 14M   \\
 opt               &     2.0M & 73M &   llvm-extract      &     420K & 23M   \\
llvm-link         &     344K & 30M  &  llvm-mc           &     916K & 20M   \\
llvm-nm           &     480K & 15M  &  objdump      &     1.2M & 22M   \\
llvm-prof         &     824K & 16M  &  llvm-ranlib       &     280K & 14M   \\
llvm-readobj      &     424K & 3.8M &   llvm-rtdyld       &     268K & 4.5M   \\
llvm-size         &     396K & 3.8M &   llvm-stress       &     604K & 14M   \\
llvm-tblgen       &     22M & 22M   & macho-dump        &     188K & 2.2M   \\

\hline
\end{tabular}
\caption{Code size comparison for LLVM code suite and Apache Server using
static libraries and shared libraries. The unit is byte.}
\label{tab:codesize}
\end{table}
The code sizes of LLVM and Apache are shown in Table~\ref{tab:codesize}.
Obviously the statically linked programs are much larger in code size than the
corresponding dynamically linked program. Increasing in code size is usually
not acceptable to industrial use, making deployment more difficult. When it
comes to concern of code size, then shared library serves well for this
purpose.

\subsection{Discussion}
Originally I tried to compile firefox~\cite{firefox} to see the code size
change. But after some attempts I found they had abandoned support of static linking 
long ago in their build scripts, so I moved to LLVM. The reason size is such huge
is because the build is a default debug build with debug information. However,
even the optimized version still occupies 1GB on disk.

Another fact is measuring performance difference between static programs and
dynamic problems are not easy. As mentioned before, the system call \emph{fork}
provides a chance to share the read only code. Thus although programs are
larger in size, it may not occupy too much memory for code regions. To compare
the performance, a set of independent programs are needed.

On the other hand, Although shared objects are extensively used in current
systems, it is not correct to say that static library is no longer required.
Some problems raise only in shared objects, like library dependency, and the
binary compatibility~\cite{dllhell}. Also, in scenario requires extreme good
performance, like high performance scientific computing, static library would
be their first choice. 
