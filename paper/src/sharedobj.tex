\section{Shared Code Page}
	\label{sec:sharedobj}
Shared code pages is an important optimization to the virtual memory. Previously
I have shown the benefits from demand allocation as an optimization, now I will
go on exploring the shared library, or shared object.  Shared code usually
appear as shared library or dynamic linking library in the system.  With shared
library, code size are largely reduced.  Moreover, the shared library is
usually place independent and can be dynamically loaded by the applications.
This enables great flexibility in design. For instance, apache~\cite{apache}
uses dynamic libraries to implement its modules, which can be loaded on demand
and built separately. In this section I compare the typical applications' code
size by linked them statically and dynamically. Performance is nearly the 
same for no matter static or dynamic linked program. I will discuss this in
\ref{sec:conc}.

\subsection{Methodology}
It's quite straightforward to measure the code size. I compiled
LLVM~\cite{llvm} and Apache~\cite{apache} to see the code size difference for
shared library version and statically linked version.

I also tried to use an apache server to exam the performance issue of dynamic
linked library. But finally I gave up, it is because the \emph{fork} system
call will make parent-children sharing the same code region, then the difference
in code size will be small. So I didn't finish the experiment about this part.

\subsection{Experiments}
 The code size from LLVM is shown in Table~\ref{table2}. Obviously the statically
 linked programs are much larger in code size than the counterpart in
 dynamically linked program. Increasing in code size is usually not acceptable to industrial use, so
 dynamic library naturally comes to our sight.

\subsection{Discussion}
Originally I would like to compile firefox~\cite{firefox} to see the code size
change. But after some attempts I found they do not support static linking in their
build scripts so I moved to LLVM.

Another fact is measuring performance difference between static programs and
dynamic problems are quite difficult. The system call \emph{fork} provides a
chance to share the read only code. Thus although programs are larger in size,
it may not occupy rally that much memory. 

Although shared objects are extensively used in current systems, it is not
correct to say that static library is no longer required. Some problems 
raised only in shared objects, like library dependency, and the binary 
compatibility~\cite{dllhell}. Also, in scenario requires extreme good performance,
like high performance scientific computing, static library would be their
first choice. 
