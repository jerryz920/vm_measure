\section{Shared Code Page}
	\label{sec:sharedobj}
Shared code pages is an important optimization to the virtual memory. Previously
I have shown the benefits from demand allocation as an optimization, now I will
go on exploring the shared library, or shared object.  Shared code usually
appear as shared library or dynamic linking library in the system.  With shared
library, code size are largely reduced.  Moreover, the shared library is
usually place independent and can be dynamically loaded by the applications.
This enables great flexibility in design. For instance, apache~\cite{apache}
uses dynamic libraries to implement its modules, which can be loaded on demand
and built separately. In this section I focus comparing the typical
applications' code size by linked them statically and dynamically.

\subsection{Methodology}
It's quite straightforward to measure the code size. I compiled
LLVM~\cite{llvm} and Apache~\cite{apache} to see the code size difference for
shared library version and statically linked version.

I also tried to use an apache server using prefork mechanism to check the
performance issues of shared library. But finally I gave up, it is because the
\emph{fork} system call will make parent-children sharing the same code region,
then the difference in code size will be small. It is possible to synthesis
some benchmark, but even if that produces bad performance for shared library, it
does not mean using shared library in real applications would yield bad
performance. Due to these considerations, I didn't finish the experiments on this. 

\subsection{Experiments}

\begin{table}[ht] \scriptsize
\begin{tabular}{|l|c|c||l|c|c|}
 \hline
 name & shared & static & name & shared & static \\
 \hline
httpd             &     1.6M & 7.3M &  bugpoint          &     3.5M & 75M   \\
clang             &     283M & 459M &   clang-check       &     135M & 152M   \\
clang-tblgen      &     6.0M & 6.0M &   llc               &     908K & 146M   \\
lli               &     844K & 103M &   llvm-ar           &     484K & 14M   \\
llvm-as           &     248K & 17M  &  bcanalyzer   &     564K & 2.6M   \\
llvm-config       &     2.1M & 2.1M &   llvm-cov          &     132K & 2.2M   \\
llvm-diff         &     1.3M & 15M  &  llvm-dis          &     384K & 14M   \\
 opt               &     2.0M & 73M &   llvm-extract      &     420K & 23M   \\
llvm-link         &     344K & 30M  &  llvm-mc           &     916K & 20M   \\
llvm-nm           &     480K & 15M  &  objdump      &     1.2M & 22M   \\
llvm-prof         &     824K & 16M  &  llvm-ranlib       &     280K & 14M   \\
llvm-readobj      &     424K & 3.8M &   llvm-rtdyld       &     268K & 4.5M   \\
llvm-size         &     396K & 3.8M &   llvm-stress       &     604K & 14M   \\
llvm-tblgen       &     22M & 22M   & macho-dump        &     188K & 2.2M   \\

\hline
\end{tabular}
\caption{Code size comparison for LLVM code suite and Apache Server using
static libraries and shared libraries. The unit is byte.}
\label{tab:codesize}
\end{table}
The code size from LLVM is shown in Table~\ref{tab:codesize}. Obviously the
statically linked programs are much larger in code size than the counterpart
in dynamically linked program. Increasing in code size is usually not
acceptable to industrial use, making deployment more difficult. To reduce
the code size and make it possible to share, shared library could serve
well for this purpose.

\subsection{Discussion}
Originally I tried to compile firefox~\cite{firefox} to see the code size
change. But after some attempts I found they had abandoned support of static linking 
long ago in their build scripts, so I moved to LLVM. The reason size is such huge
is because it is a default debug building. However, even the optimized version
still occupies 1GB on disk.

Another fact is measuring performance difference between static programs and
dynamic problems are quite difficult. As mentioned before, the system call
\emph{fork} provides a chance to share the read only code. Thus although
programs are larger in size, it may not occupy rally that much memory. 

Although shared objects are extensively used in current systems, it is not
correct to say that static library is no longer required. Some problems 
raised only in shared objects, like library dependency, and the binary 
compatibility~\cite{dllhell}. Also, in scenario requires extreme good performance,
like high performance scientific computing, static library would be their
first choice. 
