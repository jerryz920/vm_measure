\section{Experimental Environment}
	\label{sec:timing}
\subsection{Timer in Linux}
To best measure the times in experiments, I still decide to use system call
\emph{gettimeofday}. The major idea behind this is to enlarge the experiment
scale, and ammortize the overhead. There are some high resolution things like
\emph{Rdtsc} instruction on Intel's X86/64 platform. But it's hard to use, and
its behavior varies on different platforms as I tested. Though
\emph{gettimeofday} only supports measurement at ms level, we could see later,
it is enough for our experiments.

\subsection{Hardware and Software Environment}
The machine I used for testing is a x86-64 machine. The processor is Sandy
Bridge family, Intel-i5 2500K 3.3GHZ. This processor has two level private
cache, and a last level shared cache. Both L1 data cache and instruction cache
are 32KB in capacity, 8 way set associative, with 64 byte cache line. L2 cache
is 256KB, also 8 way set associative. The last level cache is 6MB. Ram size is
16GB. The operating system I chose Fedora 17, with the linux kernel version
3.3.4. Compiler version is gcc-4.7.0. I also used a performance tool called
\emph{perf}, which operates on hardware performance counter related interfaces.

