\section{Experimental Environment}
	\label{sec:timing}
\subsection{Timer in Linux}
To best measure the times in experiments, I decide to use the system call
\emph{gettimeofday}. The major idea behind this is that even if I have accurate
enough timer, it's still easy to get disturbed by system noises. To avoid these
noises, it's best to magnify the running scale, and also should run multiple
times. In this way, \emph{gettimeofday} is already enough to measure the system
behavior. There are some high resolution things like \emph{Rdtsc} instruction
on Intel's X86/64 platform. But it's hard to use, and its behavior varies on
different platforms as I tested. In my experiments, this interface works quite well.

\subsection{Hardware and Software Environment}
The machine I used for testing is a x86-64 machine. The processor is Intel-i5
2500K 3.3GHZ, family of Sandy Bridge. This processor has two level private
cache, and a last level shared cache. Both L1 data cache and instruction cache
are 32KB in capacity, 8 way set associative, with 64 byte cache line. L2 cache
is 256KB, also 8 way set associative. The last level cache is 6MB. Ram size is
16GB. The operating system I chose Fedora 17, with the linux kernel version
3.3.4. Compiler version is gcc-4.7.0. I also used a performance tool called
\emph{perf}~\cite{perf}, which operates on hardware performance counter related
interfaces, to verify some of my result and conclusions.

