\section{Huge Pages}
	\label{sec:hugepage}
Huge pages are an optimization with the demand of high performance. On latest
Intel x64-64 platform, the page size can be even 1GB. This provides good
chance for those applications suffer from virtual memory abstraction to tune
their performance, like the databases, or the host mode virtual machine monitors.
The benefits of huge pages have been long mentioned, but viewing from the
virtual memory system, there could be very time consuming to prepare these
huge pages, since huge page means continous in physical, and it will be very
difficult to find out large chunks of physical continous spaces if the system is
at busy time. In this section, I try to measure the cost to use huge pages.

\subsection{Methodology}
The experiments will contain two part, the first is the time to prepare huge
pages. This would be carried out after a busy period to observe the effect. I
choose to measure this after each Interval of section~\ref{sec:sharedobj}. Then
I try to illustrate the different timing behavior when using large pages and
small pages, by allocating the same amount of memory, and do several kinds of
walking on that.

Based on the experiments in section~\ref{sec:alloc}, it's obvious that if we
just reference a few bytes after requesting a huge page, the cost would be
larger than using a set of small pages. But this is meaningless in real
applications. So I use both sequential walking, and random access to test
the visiting time. Also, to justify the possible impact of allocation, I
also make walking happen several times: in real applications, it is not
strange to walk through a data area for several times, and the first time
page faults can be amortized in later walk.

\subsection{Experiments}
Figure~\ref{x-x} shows the cost of using small and huge pages from 2MB memory
occupation to 128MB. The stride we used to walk on the pages are from 4KB to
2MB, with multiple iterations in each walk. In the figures, huge pages performs
bad when iteration is small, and stride is relative large. With the iteration
increases, its average performance is more improved.

Random walking is more interesting. Small pages still perform better at the 
beginning, but normalized difference is smaller. The result again illustrates
the benefit of on demand allocation: sequential walking will force all pages
being brought into the systems, and that would may not be a must.

\subsection{Discussion}
Huge pages can be useful. But strange enough, in Linux normal users can get
huge page through \emph{mmap} interface using anonymous mapping, without
causing protection problem.  However, the similar access through \emph{shmget}
interface will not work, unless super user privilege is granted. I didn't 
see difference between this two approaches, hopefully it is not a security
hole.
