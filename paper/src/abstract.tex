\begin{abstract}

   Virtual memory is one of the most important subsystems inside modern
   operating systems. Although it is transparent to users, understanding the
   virtual memory can help to build better applications, especially in
   performance improvement.  In this paper, I come to four issues of virtual
   memory: TLB, User space allocator, huge pages, and optimization of code
   sharing. I use performance measurement to explore deeply about how these
   parts are working. I will explain in the paper about the methods, the
   results, and the conclustions on each separate parts.  Most of the
   experiments run as I expect, except for the TLB part. and I will try to
   illustrate why TLB measurement does not work well in some case.

%  In this paper, we find that for BTIO, through tuning the I/O configuration
%  the application performance can be improved 30\% than the default NFS
%  configuration without increasing any extra cost. As the number of I/O servers
%  of the parallel file system increases, the application performance can be
%  improved continuously. 

\end{abstract}
